
\begin{frame}{Matlab}

mad scientist

Matlab!
\begin{itemize}
\item high level dynamic domain specific language
\item focus on solving numerical/engineering problems quickly
% both productivity and speed of execution
\item over one million users
\end{itemize}

\end{frame}
\begin{frame}[fragile]{Matlab Features}
  \begin{columns}
    % code on the left
    \begin{column}[T]{5cm}
% let's look at some matlab features
% we some script here that fills some variable y with the first n
% values of some function f and plots them 
      \begin{onlyenv}<1>
        \begin{block}{plotThis.m}
          \begin{Verbatim}[commandchars=@\[\]]

  for i = 1:n
    y(i) = f(i)
  end
  plot(1:n,y)
          \end{Verbatim}
        \end{block}
          \begin{Verbatim}[commandchars=@\[\]]

>> n = 100
>> plotThis
        \end{Verbatim}
      \end{onlyenv}
% we see that there is a simple and expressive syntax, like a
% scripting language. Also notice that function calls and indexing expressions use the
% same syntax, and from a compiler point of view we don't even know
% which one is a function or a variable in this script
      \begin{onlyenv}<2>
        \begin{block}{plotThis.m}
          \begin{Verbatim}[commandchars=@\[\]]

  for i = 1:n
    @PYaG[y(i) = f(i)]
  end
  plot(1:n,y)
          \end{Verbatim}
        \end{block}
          \begin{Verbatim}[commandchars=@\[\]]

>> n = 100
>> plotThis
        \end{Verbatim}
      \end{onlyenv}
% in particular the call to f looks like a function call - but only if
% there is a function f.m on the environment path
      \begin{onlyenv}<3>
        \begin{block}{plotThis.m}
          \begin{Verbatim}[commandchars=@\[\]]

  for i = 1:n
    y(i) = @PYaG[f(i)]
  end
  plot(1:n,y)
          \end{Verbatim}
        \end{block}
          \begin{Verbatim}[commandchars=@\[\]]

>> n = 100
>> plotThis
        \end{Verbatim}
      \end{onlyenv}
% but we could also just define f to be a variable, for example a
% function handle - a function pointer of sorts - to the builtin in
% logarithm function
      \begin{onlyenv}<4>
        \begin{block}{plotThis.m}
          \begin{Verbatim}[commandchars=@\[\]]

  for i = 1:n
    y(i) = f(i)
  end
  plot(1:n,y)
          \end{Verbatim}
        \end{block}
          \begin{Verbatim}[commandchars=@\[\]]
>> n = 100
@PYaG[>> f = at log]
>> plotThis
        \end{Verbatim}
      \end{onlyenv}
% This leads to another feature: Matlab is dynamically typed
% in this case f is used as a function handle
      \begin{onlyenv}<5>
        \begin{block}{plotThis.m}
          \begin{Verbatim}[commandchars=@\[\]]

  for i = 1:n
    y(i) = f(i)
  end
  plot(1:n,y)
          \end{Verbatim}
        \end{block}
          \begin{Verbatim}[commandchars=@\[\]]
>> n = 100
@PYaG[>> f = at log]
>> plotThis
        \end{Verbatim}
      \end{onlyenv}
% in the following we build an array
% note also how there are no type declarations
      \begin{onlyenv}<6>
        \begin{block}{plotThis.m}
          \begin{Verbatim}[commandchars=@\[\]]

  for i = 1:n
    y(i) = f(i)
  end
  plot(1:n,y)
          \end{Verbatim}
        \end{block}
          \begin{Verbatim}[commandchars=@\[\]]
>> n = 100
@PYaG[>> f = (1:100).^2]
>> plotThis
        \end{Verbatim}
      \end{onlyenv}
% On a side note, notice how the colon operator builds arrays filled with
% all values from the first to the second operand in integer steps
      \begin{onlyenv}<7>
        \begin{block}{plotThis.m}
          \begin{Verbatim}[commandchars=@\[\]]

  for i = 1:n
    y(i) = f(i)
  end
  plot(1:n,y)
          \end{Verbatim}
        \end{block}
          \begin{Verbatim}[commandchars=@\[\]]
>> n = 100
>> f = @PYaG[(1:100)].^2
>> plotThis
        \end{Verbatim}
      \end{onlyenv}
% Also in the for loop, Matlab semantically creates an array of all
% integer values from 1 to n. Thus a for loop is actually a for each
% loop going over every element in an array
      \begin{onlyenv}<8>
        \begin{block}{plotThis.m}
          \begin{Verbatim}[commandchars=@\[\]]

  for i = @PYaG[1:n]
    y(i) = f(i)
  end
  plot(1:n,y)
          \end{Verbatim}
        \end{block}
          \begin{Verbatim}[commandchars=@\[\]]
>> n = 100
>> f = @PYaG[(1:100)].^2
>> plotThis
        \end{Verbatim}
      \end{onlyenv}
% We could also just put an actually array there.
% Here we actually define the iteration space in an array.
% We fill y by starting it as an empty array and appending new elements
% This illustrates another point: Every numerical variable in Matlab
% is an array. Even scalar values are actually just 1x1 arrays
      \begin{onlyenv}<9>
        \begin{block}{plotThis.m}
          \begin{Verbatim}[commandchars=@\[\]]
  y = empty
  @PYaG[for t = x]
    y = [y; f(t)]
  end
  plot(x,y)
          \end{Verbatim}
        \end{block}
          \begin{Verbatim}[commandchars=@\[\]]
@PYaG[>> x = (1:100)]
>> f = (1:100).^2
>> plotThis
        \end{Verbatim}
      \end{onlyenv}

% so instead of writing t
      \begin{onlyenv}<10>
        \begin{block}{plotThis.m}
          \begin{Verbatim}[commandchars=@\[\]]
  y = empty
  for t = x
    y = [y; f(@PYaG[t])]
  end
  plot(x,y)
          \end{Verbatim}
        \end{block}
          \begin{Verbatim}[commandchars=@\[\]]
>> x = (1:100)
>> f = (1:100).^2
>> plotThis
        \end{Verbatim}
      \end{onlyenv}

% wou could also write t(1,1) to access it as a 1x1 (2 dimensional) matrix
% Note also how all numerical operations are actually on arrays,
% either as matrix operations or elementwise operations
      \begin{onlyenv}<11>
        \begin{block}{plotThis.m}
          \begin{Verbatim}[commandchars=@\[\]]
  y = empty
  for t = x
    y = [y; f(@PYaG[t(1,1)])]
  end
  plot(x,y)
          \end{Verbatim}
        \end{block}
          \begin{Verbatim}[commandchars=@\[\]]
>> x = (1:100)
>> f = (1:100).^2 
>> plotThis
        \end{Verbatim}
      \end{onlyenv}
% like the element wise power operator here
      \begin{onlyenv}<12>
        \begin{block}{plotThis.m}
          \begin{Verbatim}[commandchars=@\[\]]
  y = empty
  for t = x
    y = [y; f(t)]
  end
  plot(x,y)
          \end{Verbatim}
        \end{block}
          \begin{Verbatim}[commandchars=@\[\]]
>> x = (1:100)
>> f = (1:100)@PYaG[.^2] 
>> plotThis
        \end{Verbatim}
      \end{onlyenv}
% Notice that we used scripts before, but we
% could as well use functions. They are defined with this funny header
% here. Unlike scripts which 'live' in the environment of the caller,
% functions have their own environment.

      \begin{onlyenv}<13>
        \begin{block}{plotThis.m}
          \begin{Verbatim}[commandchars=@\[\]]
@PYaG[function y=plotThis(f,x)]
  y = empty
  for t = x
    y = [y; f(t)]
  end
  plot(x,y)
@PYaG[end]
          \end{Verbatim}
        \end{block}
          \begin{Verbatim}[commandchars=@\[\]]
>> plotThis@PYaG[(at log,1:100)]
        \end{Verbatim}
      \end{onlyenv}
% Thus with some work we can actually figure out statically what's a
% variable and a function -- which we need to know to statically
% weave. Jesse will get back to that later.
      \begin{onlyenv}<14>
        \begin{block}{plotThis.m}
          \begin{Verbatim}[commandchars=@\[\]]
function y=plotThis(@PYaG[f],x)
  y = empty
  for t = x
    y = [y; @PYaG[f](t)]
  end
  plot(x,y)
end
          \end{Verbatim}
        \end{block}
          \begin{Verbatim}[commandchars=@\[\]]
>> plotThis(at log,1:100)
        \end{Verbatim}
      \end{onlyenv}


% matlab has a few even more dynamic features, like evaluation of
% strings, and powerful methods for introspection
      \begin{onlyenv}<15>
        \begin{block}{plotThis.m}
          \begin{Verbatim}[commandchars=@\[\]]
function y=plotThis(f,x)
  y = empty
  for t = x
    y = [y; f(t)]
  end
  plot(x,y)
end
          \end{Verbatim}
        \end{block}
          \begin{Verbatim}[commandchars=@\[\]]
@PYaG[>> eval('plotThis(at log,1:100)')]
        \end{Verbatim}
      \end{onlyenv}

    \end{column}


    % descriptions on the right
%          \item dynamically typed
%          \item everything is an array (even loops iterate over arrays)
%          \item functions vs variables confusion
%          \item functions vs scripts (functions allow you to figure
%            what is a variable)

    \begin{column}[T]{5cm}
      \begin{itemize}

        \begin{onlyenv}<1-15>
        \item simple syntax
          \begin{onlyenv}<2-4>
            \begin{itemize}
            \item but function calls and indexing look the same
            \end{itemize}         
          \end{onlyenv}
        \end{onlyenv}
        
        
        \begin{onlyenv}<5-15>
        \item dynamically typed
          \begin{onlyenv}<6>
            \begin{itemize}
            \item no type declarations
            \end{itemize}         
          \end{onlyenv}
        \end{onlyenv}

        \begin{onlyenv}<8-15>
        \item everything is an array
          \begin{onlyenv}<8-12>
            \begin{itemize}
            \item for loops are for each loops over arrays
            \end{itemize}    
          \end{onlyenv}
          \begin{onlyenv}<10-12>
            \begin{itemize}
            \item even scalars are arrays
            \end{itemize}    
          \end{onlyenv}
          \begin{onlyenv}<12>
            \begin{itemize}
            \item all operations are on arrays
            \end{itemize}    
          \end{onlyenv}
        \end{onlyenv}

        \begin{onlyenv}<13-15>
        \item functions vs scripts
          \begin{onlyenv}<13-14>
            \begin{itemize}
            \item functions provide more knolwedge
            \end{itemize}         
          \end{onlyenv}
        \end{onlyenv}

        \begin{onlyenv}<15>
        \item more dynamic features
          \begin{onlyenv}<15>
            \begin{itemize}
            \item e.g. string evaluation
            \item powerful introspection
            \end{itemize}         
          \end{onlyenv}
        \end{onlyenv}


      \end{itemize}
    \end{column}
  \end{columns}



\end{frame}





